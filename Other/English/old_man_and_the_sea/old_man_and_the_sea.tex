\documentclass[11pt]{scrartcl}
\usepackage[utf8]{inputenc}
\usepackage[normalem]{ulem}
\usepackage{soul}
\usepackage{setspace}
\usepackage{csquotes}
\usepackage{blindtext}
\usepackage[hidelinks]{hyperref}
\hypersetup{
  colorlinks   = true, %Colours links instead of ugly boxes
  urlcolor     = red, %Colour for external hyperlinks
  linkcolor    = blue, %Colour of internal links
  citecolor   = red %Colour of citations
}
\title{}
\author{}
\date{}
\doublespacing
\begin{document}

\begin{flushright}
Aarush Chaubey\\
Per. 7 AP English Literature\\
Summer Homework\\
\end{flushright}

\begin{spacing}{2}
I first met \textit{The Old Man and the Sea} as I was preparing to go on a trip to Orlando a year ago, and I decided to bring a book along to read. It's written by Ernest Hemingway(who I previously knew to be a very famous author), and Hemingway had won a Pulitzer Prize for the novella, so I thought I had made a good choice. As I started reading, however, I couldn't help but notice the sheer boredom that the book instilled. I couldn't care less about Santiago, an old Cuban fisherman who had not caught any fish for 84 consecutive days, who was rigging his lines for what seemed like 40 pages, waiting for a fish to take the bait. I couldn't understand how critics had given it such positive reviews, and, in the end, I put the book down and didn't open it again until this summer.


I had actually been reading another book before I chose, what I'll call in short, \textit{The Old Man}, but I found out that my original book was not a novel, and thus, it didn't qualify for the summer assignment. I then saw \textit{The Old Man} in Mr. Montenegro's recommended book list, so I reluctantly picked it up and started reading where I left off. This time, however, I noticed that if I paid close attention, there is a lot of emotion interwoven throughout the pages of the book - a lot of which I could relate to. At sea, Santiago was a master at his craft, solely focused on fishing even after 84 days of not catching fish. Furthermore, the novella is called \textit{The Old Man \textbf{and} the Sea}, signifying the intimate connection Santiago has with the ocean i.e the ocean and, consequently fishing, are what make up his being. This intimate connection is what really causes the old man's conflicts as he decides to travel farther out than the rest of the fishermen. A naive inference might be that the old man finds himself lost at sea, but this is not the case. He actually feels at home on the open ocean: ``A man is never lost at sea and it[Cuba] is a long island''(Hemingway 89). Santiago then hooks a large marlin who is so strong that he is pulling Santiago's skiff along with him. As the old man knows the hook has not yet firmly attached itself into the marlin's mouth, he cannot pull the marlin up yet. Instead, he has to wait for the marlin to jerk up in pain, so Santiago knows the hook is attached. Yet, for several days, the marlin just keeps swimming, and Santiago's perseverance is tested: he is forced to eat raw and drink sea water. However, this was not worrying to Santiago as, at sea, he felt at home. He even goes as far as to say, ``Fishing kills me exactly as it keeps me alive''(Hemingway 105-106). That is, the pursuit of the marlin is taking a physical toll on him, but his mental state - what keeps him driven to catching the marlin - is practically unaffected. He eventually catches the marlin, but his real troubles start once he heads back for Cuba. Because the marlin is so large, the skiff cannot support it, so Santiago has to tow the marlin on the side of his skiff, drawing the unwanted attention of sharks. As the old man fights through several shark attacks, the marlin gets eaten until all that remains is a skeleton. Basically, Santiago's pride led him out to sea and allowed him the catch of a lifetime, but it let him down in the end. This is a tone similar to that of a parable and is repeated throughout the entire novella. The Old Man is seen repeatedly praying to Mary and God; however, he also states that he isn't religious, suggesting that he(pridefully and wrongly) believes his own skill can carry him through the ordeal without any real or substantial faith. In the end of the book, he's walking up the steps of his village with ``the mast[of his boat] across his shoulder''(Hemingway 121), analogous to Christ carrying his cross. Just as Christ recognized his death, Santiago recognized his pride and his repentance is finishing his apprentice's training is what ends the book.


In addition to the ``sin of pride'' theme, there is another theme that provides insight to Hemingway's personality and writing style: a sort of early macho theme. I've only noted two women in the entire novella that provide relevant insight to the old man's character: the Mother Mary and the old man's wife. Santiago's wife is mentioned once: “Once there had been a tinted photograph of his wife on the wall but he had taken it down because it made him too lonely to see it…”(Hemingway 16), and Mary is mentioned only when Santiago is praying. Yet, both of these women are effectively ignored as Santiago can't bear to see his wife's face anymore and, as established previously, Santiago's prayers are actually empty - he instead believes he can rely on his skill to carry him through the ordeal. In fact, the being who he believes is significantly important isn't even a person at all - it's the marlin! Throughout the entire novella, Hemingway treats the marlin as his equal and views the death of the fish as inevitable, whether it be through him or some other predator. He refers to the fish, almost immediately, as male, and this ``mano e mano'' battle between the two is, effectively, most of the story. Even after their battle is over, Santiago still respects the fish as his equal, and instead of towing the marlin behind him, he ties the fish to the side of his skiff, beside him as an equal. Hemingway portrays the relationship between the marlin and the old man less like ``hunter and prey'' and more like a brotherhood.
I actually think Hemingway's attitude relates to me as well. I enjoy doing math problems that are generally harder than normal, and when I do such a problem, I treat it as a puzzle and not something that is either beyond my reach or below my level. In doing so, I take my time, sometimes being unable to solve a question for days, like Santiago does. For example, I was once attempting to solve one of the last problems in a competition, one that was considered very hard, and I thought I had a solution, just like the old man did, but it turns out my solution was wrong in the end. I had pridefully ventured too far out from sea, but when I came back home, I didn't quit solving math problems; instead, I kept on going just like Santiago did


\pagebreak
\end{spacing}
\begin{thebibliography}{}


\bibitem{book}
Hemingway, Ernest. \textit{The Old Man and the Sea}. Charles Scribner's Sons, 1952.

\bibitem{article}
Chung, Henry. ``In His Time: How Ernest Hemingway Defines and Promotes Masculinity in In Our Time.'' \textit{The University of British Columbia Arts}, Jul. 2019, \href{https://artsone.arts.ubc.ca/2019/08/02/in-his-time-how-ernest-hemingway-defines-and-promotes-masculinity-in-in-our-time}{https://artsone.arts.ubc.ca/2019/08/02/in-his-time-how-ernest-hemingway-defines-and-promotes-masculinity-in-in-our-time}

\end{thebibliography}


\end{document}
