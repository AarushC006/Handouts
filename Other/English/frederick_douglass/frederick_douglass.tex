\documentclass[11pt]{scrartcl}
\usepackage[utf8]{inputenc}
\usepackage[normalem]{ulem}
\usepackage{soul}
\usepackage{setspace}
\usepackage{csquotes}
\title{}
\author{}
\date{}
\doublespacing
\begin{document}
\begin{flushright}
Aarush Chaubey\\
Per. 2\\
Douglass Revisions\\
\end{flushright}

In the passage from Frederick Douglass' 1845 autobiography \ul{Narrative
of the Life of Frederick Douglass, an American Slave}, the author dramatizes his personal descent into hopelessness, after being physically and mentally \enquote{broken} by Mr. Covey, in order to move his reader to understand the inner turmoil preventing most slaves from ever attempting to escape to freedom.
\vspace{5mm}


Douglass opens by evoking the hopelessness and brutality felt while working for Mr. Covey through diction reflective of that describing an animal and not a human. Douglass remarked, \enquote{We were worked in all weathers. It was never too hot or too cold. Work was scarcely more the order of the day than of the night... Mr. Covey succeeded in breaking me}(1). When Douglass claims Covey breaks him, he refers to the characteristics that constituted his spirit getting systematically ripped away from him e.g. his curiosity and intelligence being \enquote{tamed by the whip.} Douglass makes it seem(correctly) that the slaves were not just being disciplined, like one would with a child, but instead tamed, like some sort of wild dog. By evoking this sense of brutality, Douglass is able to spark emotion in his audience - mostly literate, white abolitionists - to act against slavery, for Douglass was only describing what happened to him. There were many other slaves not as lucky as Douglass who had to endure torture for their entire lives. Thus, by sparking this emotion in his audience, he is able to make his audience have a look at the world through the lens of a slave.
\vspace{5mm}

In the third paragraph, Douglass employs a sequence of desperate pleas both to himself and to others, hoping to somehow rid himself of the burden of slavery. He starts by comparing himself to the boats in Chesapeake Bay of paragraph 2, \enquote{You are loosed from your moorings, and are free...O that I were free! O, why was I born a man, of whom to make a brute!...I am left in the hottest hell of unending slavery}(3). In paragraph 2, Douglass laments that he isn't free to travel the world as he pleases like the boats could, so his desperation in the next paragraph isn't just a comparison of Douglass to the sailboats in order to emphasize their freedom; rather, it emphasizes his \textit{lack of it}. Furthermore, through this comparison, Douglass allows the audience to feel what he feels: Douglass just wants to be able to live his own life without being indebted to a master, and, most probably being literate white people, the audience had never felt the way Douglass had felt. Later on in the paragraph, Douglass meticulously forms a plan for escaping, but in the end, he has a disinclination to follow through with the plan, \enquote{Besides, I am but a boy, and all boys are bound to some one}(3). This disinclination stems from Douglass being \enquote{broken} by Mr. Covey - just as a tamed dog comes back to its owner, Douglass came back to slavery. In fact, he had been just on the verge of escaping with the little wits he had that weren't stripped away from him, but the yoke of slavery pulled him back, and escape was pushed to another day. Instead of showing a simple desire to escape, Douglass shows his audience his contradictory mindset. That is, who was he outside of slavery? If his entire life was defined by not being treated as a human, was he really a human? Furthermore, this is not just Douglass-specific; in fact, Douglass shows the audience the mindsets of his brothers, sisters, friends, and family, and thus, he shows the audience the mindset of the \enquote{American Slave}.
\end{document}
