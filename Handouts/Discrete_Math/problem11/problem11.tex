\documentclass[11pt]{article}

\usepackage[utf8]{inputenc}
\usepackage[T1]{fontenc}

\usepackage[margin=1in,a4paper]{geometry}
\usepackage{graphicx}
\usepackage{enumitem}
\setlist{nosep}
\usepackage{amsmath,amssymb}

\usepackage{TwoColumnProof}

\title{Two-Column Proof Example}

%% Using example from http://www.sparknotes.com/math/geometry3/geometricproofs/section1.rhtml

\begin{document}



\section*{Given:}
Euclidean Axioms

\section*{Prove:}


\begin{TwoColumnProof}
\StatementReason{For any 6 points, the number of distinct triangles is $\binom{6}{3}$}{Choose isn't that hard to prove}


\StatementReason{The only way to have two triangles which don't intersect is having each of the 3 vertices adjacent to each other.}{Consider if they don't. Then there won't be 3 vertices on one arc of the triangle to make a non-intersecting triangle. }


\StatementReason{Thus, if we choose one group of 3 adjacent vertices, there is only one other triangle (the other 3 adjacent vertices)}{Elementary by observation}


\StatementReason{The number of ways to choose a group of 3 adjacent vertices is equal to the number of ways to choose distinct triangles, which is $\binom{6}{3}$}{For each triangle chosen, there is exactly one other triangle to be made because there are only 3 other vertices to choose}


\StatementReason{The number of ways to pick a group of 3 adjacent vertices is 6}{Count adjacent groups of 3 while going around the circle}


\StatementReason{The final probability is the number of ways to choose a group of 3 adjacent vertices divided by the number of ways to pick triangles, so $\frac{6}{\binom{6}{3}}$}{Definition of probability is what you want over what you need}

\StatementReason{Final answer is $\boxed{\frac{3}{10}}$}{Definition of division}
\end{TwoColumnProof}

\end{document}
