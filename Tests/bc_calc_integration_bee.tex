\documentclass[11pt]{scrartcl}
\usepackage[utf8]{inputenc}
%\usepackage[sexy]{evan}
\usepackage{framed}
\usepackage{amsmath}
\usepackage{float}
\DeclareMathOperator\Arg{Arg}
\DeclareMathOperator{\arcsec}{arcsec}
\DeclareMathOperator{\arccot}{arccot}
\DeclareMathOperator{\arccsc}{arccsc}
\DeclareMathOperator{\trig}{trig}
\DeclareMathOperator{\arctrig}{arctrig}
\usepackage{array}
\usepackage{enumitem}
\newcommand{\Mod}[1]{\ (\mathrm{mod}\ #1)}
\usepackage{multicol}
\usepackage{titling}
\setlength{\droptitle}{-5em}
\newcommand\floor[1]{\lfloor#1\rfloor}
\title{\underline{BC Calculus Integration Bee Qualifier}}
\author{}
\date{}


\begin{document}
\maketitle
\vspace{-25mm}
%\textbf{{\Large Doral Academy BC Calculus Integration Bee Qualifier}}

\vspace{1mm}
\noindent Answers should be in simplest form i.e. simplify fractions and expressions down to a single number. If an integral does not converge, put DNE. $\floor{x}$ is to be interpreted as the greatest integer less than or equal to $x$. Don't forget the $+C!$ 

\vspace{5mm}

\begin{enumerate}[itemsep=9mm]
    \begin{multicols}{2}
    
    \item \[\int_{69}^{420} x \,dn\] %4%
    
    \item \[\int_{0}^{\infty} e^{-ax} \,dx\] %9%
    
    \item \[\int \frac{2x}{x^2+5} \,dx\] %3%
     
    \item \[\int_{0}^{2} x^3-3x^2+3x-1 \,dx\] %5%
     
    \item \[\int_{0}^{\frac{\pi}{4}} \tan(\frac{\pi}{4}-x) \,dx\] %12%
     
    \item \[\int_{-6}^{-6} \sqrt{36-x^2}\,dx\] %6%
    
    \item \[\int \frac{1}{x^2+100} \,dx\] %1%    
    
    \item \[\int \csc(x) \,dx\]
     
    \item \[\int_{-\infty}^{\infty} \arctan(x) \,dx\] %7%
    
    \item \[\int \ln(x) \,dx\] %8%
     
    \item \[\int \frac{1}{3+25x^2} \,dx\] %14%
     
    \item \[\int_{-27}^{81} \frac{1}{x^2-7x+12} \,dx\] %11%
     
    \item \[\int_{0}^{5} \floor{x} \,dx\] %10%
     
    \item \[\int \frac{\arcsin(x)}{\sqrt{1-x^2}} \,dx\] %13%
     
    \item \[\int \frac{3x^2-10x+2}{x^3-5x^2+2x-6} \,dx\] %2%
    
    \item \[\int_{0}^{\frac{\pi}{2}} \frac{1}{1+\tan(x)} \,dx\] %15%
    \end{multicols}
    
\end{enumerate}






\end{document}
