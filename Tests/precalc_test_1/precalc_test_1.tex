\documentclass{article}
\usepackage[utf8]{inputenc}

\usepackage{framed}
\usepackage{amsmath}
\usepackage{float}
\DeclareMathOperator\Arg{Arg}
\DeclareMathOperator{\arcsec}{arcsec}
\DeclareMathOperator{\arccot}{arccot}
\DeclareMathOperator{\arccsc}{arccsc}
\DeclareMathOperator{\trig}{trig}
\DeclareMathOperator{\arctrig}{arctrig}
\DeclareMathOperator{\Real}{Re}
\DeclareMathOperator{\Imag}{Im}
\usepackage{fancybox}
\usepackage{array}
\usepackage{amsmath}
\usepackage{amssymb}
\newcommand{\Mod}[1]{\ (\mathrm{mod}\ #1)}
\title{Aarush's Super Awesome Precalculus Test}
\author{}
\date{}
\begin{document}

\maketitle
\pagebreak
\doublebox{
\parbox{\textwidth}{
       \noindent Traditional restrictions on the domains of the $\arctrig(\theta)$ functions are used. $\Real(z)$ and $\Imag(z)$ are defined as the real and imaginary components of $z$. The last 10 questions are arbitrarily ordered, so you should try and attempt \#30 even if you don't know \#21. Good luck!
       }
}
   

\vspace{3mm}
\begin{enumerate}
    \item Find the period of $2\pi\csc(4x+6)+40$
    \item Find the amplitude of $2\pi\csc(4x+6)+40$
    \item Find the phase shift of $2\pi\csc(4x+6)+40$
    \item Find the area of a regular dodecagon with circumradius 1.
    \item Find $\sin(18)$ using a trigonometric identity.
    \item Using the value you got from the previous question and the fact that $\cos(18^{\circ})$ is $\frac{\sqrt{10+2\sqrt{5}}}{4}$, find $\sin{(36^{\circ})}$.
    \item Ordóñez forgot precalculus! Help him remember the smallest angle between the planes $x+2y+3z=76$ and $4x+5y-z=14$
    \item Given an isosceles trapezoid with main diagonal of length 5 and legs of length 2, find the value of the product of the bases. 
    \item When rotating the conic $2x^2+2y^2-2xy = 69$ by 45 degrees clockwise find the sum of the coefficients on the $x^2, y^2, xy$ terms.
    
    \item Aarush rolls a die(not necessarily fair) which has sides of $\arcsin(x),\arccos(x),\\ \arctan(x), \arccsc(x),  \arcsec(x),\arccot(x)$ where $0\leq x \leq 2\pi$. Find the probability that the value Aarush rolls is undefined.
    \item Find \[\log_2\Big[{\prod_{n=1}^{44}1+\tan(n^{\circ})}\Big]\]
    \item As $n$ tends to infinity, what does the area bounded by the roots of the equation $z^n=69$ approach, where $z \in \mathbb{C}$?
     \item Tim isn't happy about the lack of American Heritage names on this test. To cheer him up, Aarush gave him a heart in the equation $r=\frac{\sin(\theta)\sqrt{|\cos(\theta)|}}{\sin(\theta) + \frac{7}{5}} - 2\sin(\theta) + 2$ where $-\pi \leq \theta \leq \pi$. Find the period of the function on  $0\leq\theta\leq2\pi$.
    \item Find the minimum value of $\Real(z)$ if \[\sum_{n=1}^{2022} |z-n|^2-|z-(n+1)|^2 \geq 0\]
    
    \item If \[\frac{\log{x}}{2022a+2b-2024c}=\frac{\log{y}}{2022c+2a-2024b}=\frac{\log{z}}{2022b+2c-2024a}\] 

    for $x,y,z,a,b,c \in \mathbb{R}$, find $xyz$.
    \item Ordóñez forgot precalculus! He needs you to find $z^{2022}+\frac{1}{z^{2022}}$ where $z+\frac{1}{z}=\sqrt{3}$.
    
 
    \textbf{The following information will be used for questions 15-18}
    
    
    \doublebox{
    \parbox{\textwidth}{The Lorentz Transformation is an often used transformation in modern physics that describes the motion of a particle(in only the x-direction for the purposes of this problem) in a separate coordinate system from a stationary frame of reference. Namely, 
    \[x' = \frac{x-vt}{\sqrt{1-\frac{v^2}{c^2}}} \]\\
    \[y' = y\]
    \[z' = z\]
    \[t' = \frac{t-\frac{vx}{c^2}}{\sqrt{1-\frac{v^2}{c^2}}}\]
    
    where $x,y,z$ describe the initial positions of the particle, $x',y',z'$ describe the positions of the particle in motion with respect to the frame of reference, and $t,t'$ describe the times with respect to the initial coordinate system and the moving coordinate system. The constant $c$ is the speed of light. For the following problems, assume traditional physical restrictions to be lifted e.g. velocities can go beyond the speed of light.
    }
    }

    \item Find \[ \lim_{v \to c} x'+vt'\]
    \item As $v$ approaches 0, if $x'+\frac{1}{t'}$ is at a minimum, find $x$ in terms of $t$, given that $x,t \geq 0 $. Hint: \textit{inequalities} 
    
    
    \item Find \[\sum_{x=1}^\infty \frac{16x}{9x^4+15x^2+16}\]
    \item Find the area bounded by $x^4-100x^2+y^4+100y^2+2x^2y^2=0$
    \newpage
    \item Find the sum of the solutions to \[\sum_{n=0}^{2022} \cos(n\theta) = -\frac{1}{2}\] for $ 0 \leq\theta \leq 2\pi$
    \item Find the number of intersections between the two functions: \[y=\sin(x) \textrm{ and } y=\frac{x}{100}\]
    \item Find \[\sum_{x=0}^\infty \arctan\Big(\frac{6x^2+2}{1+(x^2-1)^3}\Big)\]
    \item Find the period of \[\lim_{n \to \infty} \underbrace{\sin(\sin(\sin(....\sin(x))))}_{\text{composed n times}}\]
  	\item If the roots of a function $y=4x^3+2x^2-3x-1$ can be expressed as $\cos{(a)},\cos{(b)},\cos{(c)}$ where $ 0 \leq a,b,c \leq 2\pi$ are in simplest form, find $a+b+c$. Hint: \textit{substitute x= $\cos{(t)}$}
    \item Find the minimum value of \[x^4+y^6+\frac{1}{z^2}-2x^2-2y^3-\frac{6}{z}\]. Hint: $factor$
    \item Let $\gamma$ be the circumcircle of triangle $ABC$. If $K$ is constructed perpendicular to $BC$ and tangent to $\gamma$ at $A$, $D$ is constructed so that $KD\parallel AB$, and $\angle ACK = 30^{\circ}$, what is the measure of $\angle BAD$?
    \item For $x,y \in \mathbb{R}$ and $x,y \neq 0 $, if $x^2+y^2-138xy\sin^2{\theta}=0, y=ax$ where $a$ is a constant. Find $a$.
    \item Find \[\sum_{n=0}^{\infty} \arcsin{{\Big(\frac{1}{n\sqrt{n+1}-(n^2-1)\sqrt{n}}\Big)}}\]
    Hint: $\emph{Divide both numerator and denominator by}$ $\sqrt{n}+\sqrt{n-1}$.
    \item When listed in numerical order, the first, second, and third solutions described by question 21 are $\alpha, \beta, \gamma$. If $x$ has an equal chance of being either $\alpha, \beta$ or $\gamma$, find the expected value for the period of  \[\sum_{n=0}^{2022}\sin^{2n}(x)=0\]
\end{enumerate}

%4. 1/3
\end{document}
